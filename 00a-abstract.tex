
\chapter*{{\Large{\vspace{-2.3cm}Abstract}}}

\noindent \addcontentsline{toc}{chapter}{Abstract}
\fancyhead{}
\fancyfoot{}
\fancyhead[RO]{\thepage}
\fancyhead[LO]{}
\fancyhead[LE]{\thepage}
\fancyhead[RE]{\footnotesize Abstract}

\noindent The complexity of the design of radio networks has grown
with the adoption of modern standards. Therefore, the role of the
computer for the faster delivery of accurate results has become increasingly
important. In this thesis, novel methods for the planning and automatic
optimization of radio networks are developed and discussed.

The state-of-the-art metaheuristic algorithms, which compare a large
number of different network configurations, rely on model-based simulations
for the evaluation of the solution quality and the exploration of
the search space. However, current radio-network solutions, based
on snapshot simulations, have major weaknesses with respect to the
simulation time and flexibility provided. In particular, the size
of networks that can be analyzed in a feasible time is typically very
limited.

The new unified framework developed in this thesis significantly outperforms
the currently available solutions for snapshot-based, radio-network
simulations. It brings together novel and state-of-the-art parallelization
methods, in order to allow for a detailed analysis of very large networks
within an acceptable amount of time for everyday planning. This is
achieved by the parallel features of the framework, which are exploitable
on a single multi-core CPU, as well as on a network of standard PCs
with GPU devices. Clearly, the significant speedup achieved at the
simulation stage allows for an increased level of detail of the simulations,
which improves the accuracy of the results.

Increasing the performance of the simulations involved during the
objective-function evaluation is only the first step towards a practical
running-time reduction for radio-network optimization. In addition
to this, also the optimization algorithms have to be improved in terms
of speed, but not at the expense of the quality of results. In this
sense, a novel agent-based algorithm is presented and tailored to
a classic optimization problem in radio networks. The algorithm, which
is based on techniques of cellular automata and population-based metaheuristics,
shows considerable gains with respect to the size of problem instances
it may handle, as well as regarding its speed performance and solution
quality.

Another way to take advantage of the achieved speedup at the simulation
stage is to tackle problems of greater complexity. This thesis identifies
a new optimization problem in 3G radio networks that deals with soft-handover
balancing of downlink and uplink areas. Using a black-box approach,
different metaheuristic algorithms are employed for solving the problem,
the solutions of which show a substantial improvement of downlink
and uplink balance.

From the practical point of view, the automation of time-consuming,
radio-planning tasks is yet another way to exploit the benefits of
faster and more precise simulations. One such task is the configuration
of the control parameters of empirical radio-propagation models. Another
one is identified with the environmental adaptation of the signal
losses due to clutter. Both tasks are automated, using the parallel
framework as a central component of the system.

The utility and performance of the developed methods is assessed against
an enterprise, commercial tool for radio-network planning. The objective
of such comparison is to determine the performance and usage properties
of the parallel framework in everyday planning, using real-world scenarios.


