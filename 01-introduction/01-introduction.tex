
\chapter{Introduction}

% First paragraph has no indentation.

\noindent Many researchers believe the computer has become the third
method to do research, behind theory and experimentation, for both
science and engineering. Although there is no complete agreement on
the position intended for scientific computing with respect to the
other two methods, it is undeniable that computational methods are
an essential tool in most disciplines.

Scientific computing, by means of computer-science methodology, makes
possible the study of problems that are too complex to be treated
analytically, or those that are very expensive or dangerous to be
studied by direct experimentation. Real-world problems are typically
very complex systems to be directly assessed by analytical models,
and require a numerical simulation for their study. Computer simulations
provide a resource to mimic the behavior of complex systems, by numerically
evaluating a model and gathering its data to estimate their true characteristics
\cite{law2007simulation}.

A model is a simple representation of a studied problem, and one of
its purposes is to predict the effects of variations within the system.
A good model is a balance between realism and simplicity. The system
simulation, on the other hand, is the operation of the model. Its
configuration can also be changed, allowing multiple experimental
executions, something that might not be possible with the real system
it represents \cite{maria1997introduction}.

Problems that are categorized as of large size and of considerable
complexity represent a challenge because of the different involved
disciplines and the degree of difficulty of their modeling. The optimization
of 3G radio networks falls under this characterization.

A quick review of the state-of-the-art in 3G network optimization
indicates that software, providing good computational models, is a
very expensive tool for science. Moreover, since the vast majority
of this software is proprietary, it relies on closed source, formats
and protocols, which disclosure is explicitly forbidden by their licenses.
This fact creates a big hurdle to one of the key phases of scientific
methodology \cite{gauch2002scientific}: experimental reproducibility.

There is a constant growing demand for hardware resources, longer-processing
times and more memory to follow the evolution of 3G radio networks
\cite{maple2004parallel,crainic2006tackling,soldani2007autonomic}.
Fortunately, high-performance computer systems are increasingly accessible;
something made possible because of the emergence of computer clusters
and commodity hardware, capable of true parallel processing, e.g.
multi-core CPUs \cite{gorder2007multicore} and GPUs \cite{wen2011gpu}.
Moreover, the highly parallel structure present on GPUs makes them
more effective than CPUs for execution of algorithms where large blocks
of data need to be processed in parallel. Commodity GPUs have evolved
from being a graphic accelerator into a general-purpose processor.
They can achieve higher performance at lower power consumption and
lower costs when compared to conventional CPUs.

It is important to understand that the models used in scientific simulations
and engineering never offer a perfect model of the system they represent,
but only a subset of its composition and dynamics. Experimentation
and expert observation will always be essential as reference points
for understanding the studied phenomena.


\section{Radio-network optimization}

Mobile radio communications represent one of the most fast-growing
technology markets since the introduction of the second generation
(2G) mobile networks. One of the most popular implementations from
this generation is the Global System for Mobile communications (GSM)
\cite{3GPP_TR_50.099}. Its successor, the Universal Mobile Telecommunications
System (UMTS) \cite{3GPP_TR_23.101}, marks an evolution from 2G,
representing a milestone for the third generation of mobile radio
networks (3G). Since then, the increasing demand for more bandwidth
has pushed the availability of high-speed data services in order to
improve the user's experience. Once a mobile network is launched,
an important part of its operation and maintenance is monitoring the
quality characteristics and changing parameter values in order to
improve its performance.

The evolution from 2G to 3G has introduced not only the technology
needed to increase both data and voice capacity, but also a greater
complexity in terms of network planning, deployment, and configuration,
which have rendered most of the traditionally used methods to be ineffective.
In a traditional approach (i.e. manual), during the network planning
and maintenance processes, a network planning software tool would
execute the analysis, while the human would make the change decisions.
Therefore, a radio planning engineer configures network parameters
manually and the network planning tool analyzes the given configuration.
If the obtained results are not acceptable, the analysis process has
to be repeated several times, until the goal is achieved.

Modern 3G radio networks are large and many of their key parameters
are interdependent. Since an engineer is not able to cope with the
level of complexity present in such systems, the computer, along with
specialized software, guides the engineer to the most appropriate
configuration for the network. In the context of this work, we will
refer to this process as optimization.

A common limitation of the implementations of such optimization methods,
generally targeting traditional computer architectures of sequential
execution, is their inability to meet the requirements needed by real-world
mobile networks, since their computational-time complexity make them
unfeasible for practical use. When analyzing big real-world networks
with thousands of users it is necessary to reduce the execution time
of the optimization processes as much as possible, so that they are
useful for practical use.

Despite the considerable number of publications in the field of 3G
radio network optimization, of which we cite just a few for reference
purposes \cite{amaldi2007radio_planning,siomina2007minimum_pilot_power,chen2008automated,chen2009fast,gordejuela2009two,siomina2008enhancing},
most of them base their simulations on platforms for which it is not
possible to reproduce the experiments, either because the software
used is proprietary or the data is not available. We believe that
the creation of an open and standardized framework would be a great
benefit for the field of radio networks, as it would allow researchers
to compare different methods and results in an easy, fast, and objective
manner. An interesting effort in this direction is the MOMENTUM project
\cite{momentum2010}. It was created with the objective of setting
up a standardized experimental environment that would facilitate research
cooperation and ultimately improve collaboration in the field of 3G
radio networks. Unfortunately, there is no radio network simulator
included and there have been no updates in the project since 2005,
when the last data corrections were released. This situation represents
a deficit for the engineering and scientific community, since an open
simulation framework will ease the use, reproduction, and sharing
of data, related to maintenance and optimization of 3G radio networks.

Additionally, the implementation of the framework will benefit from
valuable advances in computer science and High Performance Computing
(HPC), in order to perform faster and more reliable simulations \cite{gorder2007multicore,wen2011gpu}.


\section{Organization}

The introduction provided in this chapter pretends to delimiter the
context within which the dissertation will address.

The rest if this dissertation is organized as follows.

Chapter \ref{chap:Background-and-motivation} present an overview
of some well-known optimization problems that occur during deployment
and configuration of mobile networks. A description of each optimization
problem is given, followed by a short survey of recently proposed
optimization methods. It closes by giving a conclusion regarding mobile
networks and why they are a rich source of optimization problems.

In Chapter \ref{chap:Experimental-evaluation-the-SHO-alignment-problem},
a static network simulator is used to find downlink and uplink SHO
areas. By introducing a penalty-based objective function and some
hard constraints, we formally define the problem of balancing SHO
areas in UMTS networks. The state-of-the-art mathematical model used
and the penalty scores of the objective function are set according
to the configuration and layout of a real mobile network, deployed
in Slovenia by Telekom Slovenije, d.d.. The balancing problem is then
tackled by three optimization algorithms, each of them belonging to
a different category of metaheuristics. We report and analyze the
optimization results, as well as the performance of each of the optimization
algorithms used.
