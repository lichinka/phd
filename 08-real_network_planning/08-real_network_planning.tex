
\chapter{Performance assessment within real network-planning scenarios \label{chap:08-Real-world_network_planning}}

% First paragraph has no indentation.

\noindent In this chapter, real-world, network-planning activities
on different radio networks are presented. The objective is to compare
PRATO with a radio-prediction enterprise tool in terms of quality
and performance. To this end, the engineers of the Radio Network department
at Telekom Slovenije, d.d., provided us with their radio-prediction
industrial software and guidelines to perform some typical radio-planning
activities.




\section{Measurements and simulation comparison}

We have also developed two modules for the evaluation of the accuracy
of the designed radio coverage prediction tool with respect to the
actually measured results. The first module, db.CompareResults, compares
the calculated values with the values obtained during a field measurement
campaign. The module reads each row of the text file with field measurement
data and saves the x and y coordinates, the measured signal level
and the name of the cell. The module first maps the coordinates to
the nearest coordinates in the database (as they do not necessarily
coincide), finds the corresponding row in the database and then extracts
the received signal level for the required cell. The data for the
measured and calculated signal level, their difference in decibels
along with the location, the used path loss prediction model and the
cell name are finally written to the output text file (Table 3).

The second module, r.EvaluateSimulations, compares the field measurements
and the strongest simulation signal levels, neglecting the information
about the serving cell. The module output is a textual file including
the same fields as the output of the db.CompareResults module. The
module enables comparison of field measurements with simulation results
calculated with the designed radio coverage prediction tool or the
commercial tool TEMS. Because of different input raster files (a GRASS
raster file includes values for maximal received power while a TEMS
raster file contains path loss values) additional selection is done
with the \textquotedblleft{}GRASS MaxPower raster\textquotedblright{}.
In the case of the TEMS input raster file, an average transmitted
power value must be entered for the receive power calculation.


\section{Coverage-prediction performance analysis}

The performance and accuracy of the developed modules for radio signal
coverage prediction was investigated by comparing simulation results
and field measurements. The reference values were obtained by comparing
field measurements with simulation results acquired from the professional
radio signal coverage prediction program TEMS. In both simulation
tools we used the modification of the Okumura-Hata propagation model
{[}23{]}.

The performance of the new software package was investigated for different
types of networks (GSM, UMTS) and terrains (hilly and almost flat
rural, urban, and suburban). The evaluation of the developed software
for different frequency bands is presented first, followed by an analysis
for a different terrain type.

The accuracy of the GRASS prediction software can be verified from
the charts on Fig. 6, 7 and 8. On the left side, the charts comparing
the measurements and calculations with the GRASS radio coverage prediction
software are depicted, while graphs showing the comparison between
the measurements and calculations with the TEMS software package are
on the right.

Fig. 6 and 7 are presenting the simulation results from both tools
and the field measurements in suburban environment for 900MHz and
2040MHz. Received power charts clearly shows that the simulation results
match the measured values rather well. Slightly better agreement can
be perceived in the 900MHz frequency band (Fig. 7). The deviation
among the measurements and simulations for both software applications
is depicted in the second raw of diagrams in Fig. 6 and 7. It is evident
that the difference between the diagrams on the left- and on the right-hand
side for both frequencies is minor. Thus, it can be concluded that
the results from the developed radio coverage tool are comparable
with the results from the TEMS application and are independent from
the used frequency band.

Additional analyses were done on different terrain types. The analyses
for the flat rural environment are depicted in Fig. 8. The curves
on the charts showing the difference between the measurements and
simulations for both software applications have similar course. This
confirms applicability of the developed software also for arbitrary
terrain types.

The developed radio coverage software gives similar results as the
professional TEMS software irrespective of the operational frequency
or chosen terrain type. The computed values are comparable also for
different distances between the base station and the receiver. Some
negligible differences between the results originate from the fact
that the implemented path loss model in the TEMS software used in
the simulation is not entirely available and thus cannot be realized
in the GRASS software in a completely identical way.

Additionally, execution performance of the developed modules was evaluated
in terms of the required processing times on our system (processor
Intel Core2 Quad CPU 2,66GHz, disk WD2500KS, OS Linux RHEL5). The
simulated configuration included eight transmission antennas on four
locations (base stations), therefore requiring four model and eight
sector computations. Two different geographic regions were used: a
small one, \textquotedblleft{}Ljutomer\textquotedblright{}, encompassing
all transmission locations (15 x 13km, resolution 25m) and a large
one \textquotedblleft{}Slovenia\textquotedblright{} (whole Slovenia,
285 x 185km, resolution 100m). The effective transmission radius was
limited to 10km. The results for single core execution are given in
Table 4. The hataDEM model was not simulated for the whole Slovenia
region since its clutter map was not available to us. The r.MaxPower
module was not run with DBF database on the whole Slovenia region
since the internal GRASS DBF processing is very memory inefficient,
keeping the whole database in the main memory and hence running out
of memory for large regions.


\subsubsection{Performance analysis}

In the following, in order to confirm the benefits of PRATO as a parallel
simulation framework for radio networks, we contrast the running time
of a serial implementation deployed in one host against PRATO over
multiple processors.

We have adapted the serial implementation for execution in a feedback
loop together with the DASA. To this end, the implemented code dealing
with input-data load and set up has been moved outside the loop, so
that the objective-function evaluations alone, including the radio-propagation
predictions, are executed in each iteration of the optimization algorithm.
This way, only the running time is compared as a performance metric,
excluding load time from the analysis, since both implementations
use the same input-data sets.

The running-time measurements for the adapted serial implementation
were collected on a host featuring an i7-2600K quad-core processor.
The measurements for PRATO were taken over multiple processors, with
only the master process running on an i7-2600K CPU. Please refer to
Section \ref{sub:05-Experimental_environment} for a reference of
the hardware configuration and process layout used during the experimental
simulations.

We have performed 300 independent time-measurements of the adapted
serial implementation. From this set, we pick the minimum (fastest)
running time of a single radio-propagation prediction (0.526333~sec)
in order to linearly estimate the running time of the complete optimization
process for each of the three test networks. Time-measurement gathering
for the running time of PRATO over multiple processors has been performed
during the simulation runs presented in the previous section, from
which the minimum time is used for comparison. All gathered running-time
measurements and estimations for the adapted serial implementation
are depicted in Table \ref{tab:Performance-analysis-serial}. We may
observe that the estimated running times of the optimization process
using the adapted serial implementation are unfeasible for practical
use. Specifically, the smallest network, Net$_{3}$, shows an estimated
running time of over 42 hours, while the biggest network instance,
Net$_{2}$, raises over 2,200 hours! 

The best running times of the parallel implementation with multiple
processes deployed on the computer cluster are shown in Table \ref{tab:Performance-analysis-parallel}.
The speed-up factors for each test network, ranging from 4.50 to 119.58,
are also depicted in the third row. These values are an estimation,
since they are based on the estimated running times of the adapted
serial version. Despite this, we may clearly notice how the estimated
efficiency values, calculated as the speed-up factor over the number
of processes, increase with the problem size, thus improving the exploitation
of the computing resources of the cluster. This behavior corresponds
with the speed-up measurements presented in Figure \ref{fig:strong_scalability_speedup}.
Nevertheless, the obtained speed-up factors are substantial, lowering
the running time of the optimization process from 84.2 to 8.7 hours
for Net$_{1}$, from 2,280.8 hours to 19.1 hours for the biggest instance,
Net$_{2}$, and from 42.1 to 9.3 hours for Net$_{3}$, making the
presented optimization approach feasible for practical use in real-world
radio networks.

\begin{table}
\caption{Measured and estimated running times of the clutter-optimization process
using the adapted serial implementation. Times are depicted in seconds
for the three test networks. \label{tab:Performance-analysis-serial}}


{\footnotesize{\centering}}{\footnotesize \par}

\begin{tabular}{>{\raggedright}p{2.5cm}cc>{\centering}p{1.5cm}}
\cline{2-4} 
 & Net$_{1}$ & Net$_{2}$ & Net$_{3}$\tabularnewline
\hline 
{\footnotesize{Fastest radio-coverage prediction}} & {\footnotesize{0.526333}} & {\footnotesize{0.526333}} & {\footnotesize{0.526333}}\tabularnewline
\hline 
{\footnotesize{Estimated objective-function evaluation time}} & {\footnotesize{6.315996}} & {\footnotesize{68.423290}} & {\footnotesize{3.157998}}\tabularnewline
\hline 
{\footnotesize{Estimated optimization-process time}} & {\footnotesize{3.031678$\cdot10^{5}$}} & {\footnotesize{8.210794$\cdot10^{6}$}} & {\footnotesize{1.515839$\cdot10^{5}$}}\tabularnewline
\hline 
\end{tabular}
\end{table}


\begin{table}
\caption{Measured running times (in seconds) of the clutter-optimization process
using PRATO over multiple processors. The number of deployed worker
processes and speed-up factors are also depicted for the three test
networks. \label{tab:Performance-analysis-parallel}}


{\footnotesize{\centering}}{\footnotesize \par}

\begin{tabular}{>{\raggedright}p{2.5cm}cc>{\centering}p{1.5cm}}
\cline{2-4} 
 & Net$_{1}$ & Net$_{2}$ & Net$_{3}$\tabularnewline
\hline 
{\footnotesize{Fastest optimization-process time}} & {\footnotesize{3.154711$\cdot10^{4}$}} & {\footnotesize{6.866361$\cdot10^{4}$}} & {\footnotesize{3.368531$\cdot10^{4}$}}\tabularnewline
\hline 
{\footnotesize{Number of worker processes}} & {\footnotesize{12}} & {\footnotesize{130}} & {\footnotesize{6}}\tabularnewline
\hline 
{\footnotesize{Estimated speed-up}} & {\footnotesize{9.61x}} & {\footnotesize{119.58x}} & {\footnotesize{4.50x}}\tabularnewline
\hline 
{\footnotesize{Estimated efficiency}} & {\footnotesize{0.80}} & {\footnotesize{0.92}} & {\footnotesize{0.75}}\tabularnewline
\hline 
\end{tabular}
\end{table}



\section{Summary}

Precise and efficient planning of the wireless telecommunication systems
requires efficient and exact radio signal coverage calculations. The
high price and limited functionalities of the existing professional
network planning tools compels to look for alternative solutions.
The needs can be fulfilled using an open-source system which gives
possibilities to improve the existing models based on measurements,
or to develop entirely new path loss prediction models. This paper
presented a radio signal coverage prediction software tool developed
for the open-source GRASS system. After a short introduction of the
GRASS GIS system, a detailed description of the coverage prediction
software was given. The tool enables a high level of flexibility and
adaptability. It is composed of several GRASS modules for path loss
calculation, a sectorization module, a module for radio signal coverage
calculation, and additional modules for preparing input data and analyzing
simulation results. Modules can be used individually or through the
r.radcov module, written in Python, which interconnects individual
modules into a complete radio signal propagation software package.
At the end, the developed software was evaluated by comparing with
the field measurements and simulation results obtained from a professional
software application. The radio signal coverage prediction software
implementation was quite straightforward, as API is well developed
and documented. The set of built-in C functions is adequate. The possibility
to study parts of the already implemented code is also very helpful.
Extensive performance analyses showed satisfactory results. Compared
to a professional network planning tool, the computation speed is
slightly lower while the result accuracy is completely comparable
irrespective of the terrain type or operational frequency. For better
agreement between simulations and measurements, additional model tuning
will be performed. In our future work, we also plan to expand the
functionalities of the developed software package and build additional
path loss modules for the urban and hilly rural environments that
will also include the elements of ray tracing techniques and additional
environment data {[}29{]}. The achievement made so far represents
a strong base for future work and is interesting both from the point
of view of researchers as well as network developers. The whole source
code of the radio signal coverage prediction tool together with detailed
instructions will be publicly available at http://commsys.ijs.si/en/software/grassradiocoverage
tool. The tool can be freely used, modified and upgraded with new
path loss modules.
