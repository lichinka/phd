
\chapter{Conclusion and Further Work \label{chap:Conclusion}}

Fast and accurate performance evaluation is of essential importance
for the planning and optimization of mobile radio networks. In this
thesis, several high-performance and novel methods for the analysis
and optimization of radio networks were developed and discussed. Some
of them even outperform state-of-the-art methods to which they were
compared in terms of result accuracy and instance-size complexity.

The snapshot analysis is best suited for the detailed performance
evaluation of radio networks. The performance of a snapshot method
is mainly influenced by two factors. First, the performance of the
method for modeling a single snapshot and, second, the quality level
of the estimate applied for obtaining the interest figures from the
individual snapshot results. Both topics were comprehensively addressed
in this thesis.

While providing detailed and accurate results for various applications
in network planning, the snapshot analysis is too time-consuming to
be applied in radio-network optimization, where typically a large
number of different configurations need to be compared in a shorter
time-frame. To this end, a parallel framework for radio-network planning
was presented in this thesis. The framework is very flexible in terms
of air-interface modeling, e.g., different QoS schemes, and is of
significantly higher computational-time performance compared with
any currently available solution known to the author. Moreover, it
incorporates multi-GPU support and the corresponding parallel-programming
techniques that are required to exploit the computational power of
such hardware. The performance gain results mainly due to the combination
of a novel parallel-programming approach and some state-of-the-art
methods. The achieved performance improvement excels in radio-network
optimization environments. Further research in this field will include
abstracting the introduced master-worker-database principle into a
multi-purpose parallel framework such as Charm++~\cite{Kale-The_Charm_Approach:2013},
which provides a functionality for overlapping execution and communication,
as well as fault tolerance.

When applying automatic network optimization, a fast and accurate
performance analysis is of crucial importance. However, an overview
of the related literature shown that this fact seems to have been
only partially taken into consideration in the design of several optimization
methods for radio networks. A common approach is to apply a detailed
and too time-consuming method for evaluating different candidate solutions,
resulting in an unacceptable computational-time complexity for medium
to large-sized problem instances. The opposite approach, which is
even less satisfactory, is to build fast but inaccurate models, the
results of which do not sufficiently correlate with reality. To this
end, the quality and speed performance of the presented framework
was empirically verified with an industrial software tool for radio-network
planning. The results clearly show a very good agreement in terms
of accuracy of the radio-propagation predictions, compared to those
obtained with the commercial tool. Additionally, the results demonstrated
the performance advantage of the framework compared to the computational
time of the enterprise software. It is important to note that these
results also apply for an application in everyday network planning.
However, to validate the complete array of radio-planning activities
the framework can handle further research is required.

Increasing the performance of the simulations involved during the
objective-function evaluation is only the first step towards a practical
running-time reduction for radio-network optimization. In this sense,
the performance of the novel agent-based algorithm presented in this
thesis was tested while solving the service-coverage problem in radio
networks. The results show significant gains with respect to the size
of problem instances, as well as regarding its speed performance and
solution quality, even outperforming a state-of-the-art method, to
which it was compared. Further research will include experimentation
with different parameters and optimization problems, in order to gain
better understanding of the dynamics that guide the algorithm through
the search space of the problem.

The new optimization problem for 3G radio networks identified in this
thesis deals with SHO balancing of downlink and uplink areas. The
problem was tackled by three different metaheuristic algorithms, the
solutions of which show a substantial improvement of downlink and
uplink balance. A challenge for future work is to evaluate this optimization
problem in a dynamic context. This requires using a full-stack simulation
tool that includes dynamic effects, such as fast power control.

The clutter-loss optimization method developed in this thesis makes
use of the tools discussed above, providing a faster, more accurate
and simpler method that replaces a manual approach. It applies a metaheuristic
algorithm which has been tailored for the special requirements of
the automatic clutter-loss optimization. The presented results make
the benefits of an automated-optimization approach evident. Furthermore,
in the context of other radio-coverage planning activities carried
out at the Radio Network department of Telekom Slovenije, d.d., supplementary
testing of the framework as a coverage-planning tool is currently
being conducted.

PRATO, the radio-planning framework presented in this thesis, is a
free and open-source software. For this reason, it can be readily
modified and extended to support, for example, other propagation models
and post-processing algorithms. This characteristic provides it with
a clear advantage when compared to commercial and closed-source tools.
The source code is available for download from the author's home page,
http://cs.ijs.si/benedicic/.

The constantly improving performance of computer hardware might allow
an even more detailed analysis of networks in the near future. In
this sense, a major challenge that will require further research is
to extend the presented models and algorithms, e.g., by incorporating
user mobility and traffic dynamics. This is especially important due
to different existing and emerging mobile technologies. Indeed, the
co-existence of multiple technologies will motivate an increasing
interest about the combined analysis and optimization of different
technologies and their mutual dependency.


\section{Scientific contributions}

The work in this thesis has led to the following original contributions
to science:
\begin{enumerate}
\item Design and development of a unified framework for radio-network planning
and optimization, incorporating the second and third contribution
items, as well as experimentation on real-world radio networks. The
experimentation includes a comparison with a commercial tool that
is currently being used for real radio-network planning. Moreover,
the parallel implementation of the framework exploits the computing
resources of computer clusters and GPUs.
\item Proposal of a new approach for parallel programming that combines
a classic master-worker method with an external database. The proposed
approach improves the scalability of the classic master-worker paradigm
by preventing the master process to become the bottleneck of a parallel
system.
\item Quality improvement of radio-propagation predictions by applying metaheuristic
optimization to the parameters of radio-propagation models. This technique
enables the adaptation of radio-propagation models to the local environment
over which a radio network is deployed, as well as the automatic optimization
of signal losses due to clutter.
\item A new algorithm, based on autonomous agents, to tackle the service-coverage
problem in radio networks. The algorithm deals with problem instances
that are out-of-reach of other compared state-of-the-art techniques
used as reference. It also reaches good quality of solutions if compared
to classic network-planning techniques. The proposed approach is especially
suitable for optimizing large radio networks.
\item Identification and formalization of a new optimization problem in
3G radio networks that deals with SHO alignment of downlink and uplink
areas. By solving this problem, network malfunctioning is avoided
in areas where there is SHO capability in the uplink, but none in
the downlink. So far, this problem has not been formalized nor tackled
by means of automatic optimization.\end{enumerate}

