
\chapter{Conclusion and further work \label{chap:Conclusion}}

The methods and tools introduced in this thesis have a direct correlation
with the simulation and optimization of radio networks in general
and with those of the 3G and 4G in particular. We will introduce the
steps necessary to mitigate the problem of experimental reproducibility
found in most published works, by simplifying setup, execution, and
sharing of experimental results. With the development of the framework,
we will evaluate and assess the possibilities it offers as a support
system for simulation and optimization problems of radio networks.
By including parallel programming techniques for computer clusters
and GPUs, we intend to go beyond the classical methodology provided
by previous works, taking advantage of the inherent parallelism of
some optimization techniques. We will also be looking into the application
of many advances in HPC that should provide the framework with the
computing power needed to improve the simulation process, thus enhancing
its scalability to support real-world 3G radio networks. Finally,
since we believe that this work will only become truly productive
through the cooperation and long-term development of the scientific
and engineering community, we will be releasing the source code, algorithms,
documentation, and data to the public domain. This way, anyone will
be able to use and to extend the framework for their own needs. Encouraging
cooperation and sharing of experimentation-related tools and data
should be a common goal from which everyone will benefit.

We have presented an open-source simulation framework for coverage-planning
and optimization of radio networks (PRATO)%
\footnote{The source code is available for download from the corresponding author's
home page, http://csd.ijs.si/benedicic/%
}.

Encouraged by the favorable results, further work will include abstracting
the introduced MWD principle into a multi-purpose parallel framework
such as Charm++ \cite{Kale-The_Charm_Approach:2013}, which provides
a functionality for overlapping execution and communication, as well
as fault tolerance.

Encouraged by the favorable results, further work will include abstracting
some of the introduced principles and methodology into a multi-purpose
library for parallel simulation of radio-coverage predictions, which
shall be published as free and open source software. The aim of such
tool is to ultimately validate the suitability and usefulness of the
presented framework.

Furthermore, in the context of radio-coverage planning activities
carried out at the Radio Network department of Telekom Slovenije,
d.d., supplementary testing of PRATO as a coverage-planning tool is
currently being conducted. So far, the performed predictions show
evidence of faster and high-quality results when compared with well-established
industrial software tools.

In addition, as PRATO is also a free and open-source software project%
\footnote{The source code is available for download from http://cs.ijs.si/benedicic/%
}, it can be readily modified and extended to support, for example,
other propagation models and post-processing algorithms. This characteristic
provides it with a clear advantage when compared to commercial and
closed-source tools.

Comparison of our experimental results with other algorithms dealing
with the same and similar problems would be useful. However, this
task is not straightforward, since the results of several works (e.g.
\cite{Gerdenitsch_PhD:2004,Turke_Advanced.site.configuration.techniques:2005})
depend on black-box evaluations, making experimental association very
difficult, if possible at all. 

All in all, we consider that the present work provides a robust foundation
for future work on grid-based metaheuristics with expensive objective-function
evaluation.

In future work, we will consider further analysis of our parallel-agent
approach, including experimentation with different parameters, in
order to gain better understanding of the dynamics leading the metaheuristic
during the search process. Multi-GPU environments present an interesting
possibility, where evaluator(s) and worker agents are run on separate
GPU devices.

To further improve the presented results, dynamic effects, such as
fast power control, should be included in the simulations, particularly
for recognizing dynamic functionality like SHO in a WCDMA mobile system.
Another extension of the current work is to incorporate antenna tilt
as an additional objective of the optimization process. This should
certainly include experimentation with models and algorithms that
support multiobjective optimization.

It is important to note that some methods proposed in this thesis
have been particularly designed for problems emerging in radio planning
of 3G networks. Despite this, they may be adapted to other standards,
e.g. GSM or LTE, without lose of generality. Moreover, some of them
may even be applied in other research areas, e.g., parallelization
techniques in the area of greographical information sciences.

As it was mentioned in the introduction, it is important to understand
that the models used in scientific simulations and engineering never
offer a perfect model of the system they represent, but only a subset
of its composition and dynamics. For this reason, experimentation
and expert observation will always be essential as reference points
for understanding the studied phenomena. Consequently, problems categorized
as of large size and of considerable complexity represent a challenge,
because of the different involved disciplines and the degree of difficulty
of their modeling. Radio networks, in particular those of the third
and fourth generations, fall under this characterization.

It is very important to develop a feeling for the properties, advantages
and drawbacks of the respective methods. Moreover, the recommendations
of radio experts regarding the interpretation of the solutions and
the feedback from everyday network operation are an essential input
for establishing high-quality optimization methods. In this sense,
and based on our own experience, the expert's advice is irreplaceable
and a most valuable contribution to the research work.


\section{Scientific contributions}



The work in this thesis has led to the following original contributions
to science: 

1. A unied anomalous and suspicious behavior detection framework:
We proposed a unied framework for detection of anomalous and suspicious
behavior that can be observed from complex, spatio-temporal sequential
data generated by an agent moving in a physical environment. The framework
incorporates several components to address the main challenges and
is demonstrated empirically in three studies. 

2. Contribution to anomalous and suspicious behavior detection: We
gave the rst clear problem denition and established a theoretical
framework for anomalous and suspicious behavior detection from agent
traces to show how to optimally perform detection. We discussed why
detection error is often inevitable and proved the lower error bound.
We further provided several heuristic approaches that either estimated
distributions required to perform detection or directly rank the behavior
signatures using machine-learning approaches. 



The contributions of this thesis to the fields of telecommunications
and computer sciences include the following:
\begin{itemize}
\item State-of-the-art overview of optimization methods for 3G radio networks.
\item Design and development of a framework that provides an open environment
for radio network simulations, implemented for execution on computer
clusters and GPUs. The framework will allow the scientific community
to share a common domain to run the simulations needed by modern optimization
methods, since most currently available simulation tools are proprietary
and therefore unsuitable for experimental reproducibility.
\item Improvement of quality and speed of renowned mathematical models,
used for radio propagation predictions, by applying parameter optimization
and parallelization techniques. The expected speed improvement should
be of at least one order of magnitude.
\item Proposal of a new algorithm, based on autonomous agents, to solve
the service coverage problem. The solved problem instances should
be of bigger size than ever solved in the literature and reach equal
or better quality thereof. This should make our approach applicable
for large real-world problem instances and data sets.
\item Identification of a new optimization problem in 3G radio networks
that deals with soft-handover alignment of downlink and uplink areas.
By solving this problem, we should avoid abnormal network functioning
in areas where there is soft-handover capability in the uplink, but
none in the downlink. So far this problem has been solved manually
by radio experts.
\item Empirical comparison of the proposed metaheuristic algorithm against
the existing state-of-the-art optimization algorithms on the soft-handover
alignment problem.\end{itemize}

