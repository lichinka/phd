
\chapter*{{\Large{\vspace{-2.3cm}Appendix B: Biography}}}

\noindent \addcontentsline{toc}{chapter}{Appendix B: Biography}
\fancyhead[LO]{}
\fancyhead[RE]{\footnotesize Appendix B: Biography}

\noindent % First paragraph has no indentation.

\noindent Lucas Benedi\v{c}i\v{c} was born on December 12, 1977 in
Buenos Aires, Argentina. He was a student at the University of Buenos
Aires, where he completed the first 4 years of the undergraduate programme
in computer sciences, at the Faculty of Exact and Natural Sciences.

After moving to Slovenia in 2002, he enrolled into the University
of Ljubljana, Faculty of Computer and Information Science, where he
graduated in 2007 with the work titled ``Application development
using open-source technologies''.

In September 2007, he started his postgraduate studies at the University
of Primorska, Faculty of Mathematics, Natural Sciences and Information
Technologies, after enrolling into the Masters study programme of
computer sciences. He graduated in 2009 with the work titled ``Optimization
of Common Pilot Channels in UMTS Networks''.

Towards the end of 2007, he was awarded a scholarship for continuing
his postgraduate studies in Slovenia. The award was received after
winning a problem-solving contest, organized by Halcom, d.d. In the
fall of 2009, he enrolled in the PhD program entitled ``Information
and Communication Technologies'' under the supervision of Assoc.~Prof.~Peter
Koro\v{s}ec and the co-supervision of Assist.~Prof.~Toma\v{z}
Javornik, at the Jo\v{z}ef Stefan International Postgraduate School
in Ljubljana, Slovenia.

In November 2009, he became a junior researcher, partially funded
by the European Union through the European Social Fund. During this
period he held working positions at Telekom Slovenije, d.d., in research
and development within the Radio Network Department, and at the Jo\v{z}ef
Stefan Institute, in the Computer Systems Department.

In 2011, he attended and successfully completed the summer institute
``Scientific Computing in the Americas: the challenge of massive
parallelism'', organized by the Pan-American Advanced Studies Institute,
Boston University. The focus of the institute was on advanced theoretical
topics and programming of massively parallel processors.

In early 2012, he received an invitation to work as a visiting researcher
at the Nagasaki Advanced Computer Center (NACC) at the University
of Nagasaki, Japan. His joint research work at NACC was completed
by the end of 2012.

Over the past few years, he has presented his research work at several
international conferences and workshops in the areas of evolutionary
computation, optimization and parallel computing.

His research interests include combinatorial and numerical optimization,
parallel and GPU computing, and metaheuristic algorithms.
