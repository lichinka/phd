%!TEX root = thesis.tex

%\begin{preliminary}{Povzetek} 

  \cleardoublepage
  \thispagestyle{fancy}
  \fancyhead{}
  \fancyfoot{}
  \fancyhead[RO]{\thepage}
  \fancyhead[LO]{}
  \fancyhead[LE]{\thepage}
  \fancyhead[RE]{\footnotesize Povzetek}
  \vspace{-2.4cm} 
  {\noindent \Large \bfseries Povzetek}
  \vspace{1.5cm}

\vspace{-1mm}
\noindent
Aplikacije na podro{\v c}jih pametnih okolij, video nadzora, interakcije {\v c}lovek-robot in ambientalno podprtega {\v z}ivljenja obi{\v c}ajno vklju{\v c}ujejo problem u{\v c}enja vzorcev obna{\v s}anja agenta iz senzorskih podatkov. Odklonsko obna{\v s}anje je vzorec v podatkih, ki se bodisi ne ujema s pri{\v c}akovanim obna{\v s}anjem, kar ustreza nenavadnemu obna{\v s}anju, bodisi se ujema s predhodno definiranim neza{\v z}elenim obna{\v s}anjem, kar ustreza sumljivemu obna{\v s}anju. Pri{\v c}ujo{\v c}a disertacija se osredoto{\v c}a na detekcijo vzorcev, ki lahko predstavljajo varnostno gro{\v z}njo, zdravstveni problem ali kakr{\v s}nokoli drugo tveganje, povezano z obna{\v s}anjem agenta. 
%Primeri vklju{\v c}ujejo detekcijo sumljivih potnikov na letali{\v s}{\v c}u, ki se izogibajo stiku z varnostnim osebjem, in napadalca, ki sku{\v s}a pridobiti dostop do visoko varovane vstopne to{\v c}ke z zlorabo identitete.


Pri aplikacijah v realnem {\v z}ivljenju se soo{\v c}amo s {\v s}tevilnimi izzivi. Raziskave na podro{\v c}ju razpoznavanja planov so predpostavile, da so osnovne akcije agenta podane ali pa jih je mogo{\v c}e enostavno pridobiti, medtem ko mnoge aplikacije v realnem {\v z}ivljenju zahtevajo prepoznavanje akcij iz surovih senzorskih podatkov. Drugi izziv je kako predstaviti zapleteno, nestrukturirano obna{\v s}anje ljudi, ki ne sledijo vnaprej dolo{\v c}enim vzorcem. Tretji izziv predstavlja dejstvo, da se odklonsko obna{\v s}anje lahko odra{\v z}a na razli{\v c}nih {\v c}asovnih intervalih in preko razli{\v c}nih zaznavnih vhodov, pri {\v c}emer se poraja vpra{\v s}anje kako zdru{\v z}evati razli{\v c}ne {\v c}asovne intervale in zaznavne vhode pri pridobivanju zanesljive ocene obna{\v s}anja. In nenazadnje, v mnogih domenah je prisotno obna{\v s}anje, kjer iz posameznega zaznanega dogodka ni mogo{\v c}e sklepati ali je obna{\v s}anje odklonsko ali ne, zato je potrebno vpeljati pristop, ki lahko kopi{\v c}i ocene obna{\v s}anja v dalj{\v s}ih {\v c}asovnih obdobjih. 


V pri{\v c}ujo{\v c}i disertaciji predstavimo enoten okvir za analizo oba{\v s}anja agenta na podlagi predhodnega znanja in zunanjih opa{\v z}anj. Namenjen je odkrivanju odklonskega obna{\v s}anja agentov, ne glede na to ali je predmet opazovanja {\v c}lovek, programski agent ali robot. Disertacija najprej predstavi cevovod za razpoznavanje aktivnosti, ki vklju{\v c}uje odstranjevanje {\v s}uma, izdelavo zna{\v c}ilk, identifikacijo aktivnosti in izravnavanje {\v s}uma pri razpoznavanju. V nadaljevanju opi{\v s}e novo predstavitev, poimenovano prostorsko-akcijska matrika, namenjeno analizi obna{\v s}anja. Z matriko je mogo{\v c}e z uporabo prostorsko-akcijskih zna{\v c}ilk opisati dinamiko obna{\v s}anja v dolo{\v c}enem {\v c}asovnem obdobju ter grafi{\v c}no ponazoriti primerjavo med razli{\v c}nimi vzorci obna{\v s}anja. Predstavljen je postopek, ki s pomo{\v c}jo analize glavnih komponent zmanj{\v s}a dimenzije matrike ter poda njene zna{\v c}ilke. V disertaciji se nato osredoto{\v c}imo na definicijo problema in vzpostavimo formalni okvir za detekcijo nenavadnega in sumljivega obna{\v s}anja. Na podlagi formalnega okvira razlo{\v z}imo, zakaj je napaka pri detekciji obi{\v c}ajno neizogibna, podamo dokaz za spodnjo mejo napake in predstavimo {\v s}tevilne pribli{\v z}ne metode, ki bodisi neposredno ocenijo porazdelitve, potrebne za detekcijo, bodisi razvrstijo vzorce obna{\v s}anja z uporabo strojnega u{\v c}enja. Formalni okvir je nato raz{\v s}irjen z mo{\v z}nostjo zaznavanja odklonskega obna{\v s}anja v dalj{\v s}em {\v c}asovnem obdobju, kjer posamezen dogodek ne zadostuje za odlo{\v c}itev. Disertacija poda pogoje, ki jih mora detektor izpolnjevati, in predstavi nov pristop poimenovan detektor F-UPR, ki posplo{\v s}i razpoznavanje planov na podlagi koristnosti s poljubnimi funkcijami koristnosti. Uporabo enotnega okvira za analizo obna{\v s}anja agenta predstavimo v treh empiri{\v c}nih {\v s}tudijah. Prva {\v s}tudija se nana{\v s}a na detekcijo obna{\v s}anja, ki nakazuje poslab{\v s}anje zdravstvenega stanja starej{\v s}ega posameznika, medtem ko se druga ukvarja z detekcijo sumljivih potnikov na simuliranem letali{\v s}kem terminalu. Tretja {\v s}tudija zadeva preverjanje identitete vstopajo{\v c}e osebe v visoko varovanih kontrolnih to{\v c}kah vstopa.

%\end{preliminary} 
