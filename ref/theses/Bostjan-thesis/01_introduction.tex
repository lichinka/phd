%!TEX root = thesis.tex

\chapter{Introduction}
\label{chap:introduction}

% Chapter intro...
The problem of learning behavior patterns from sensor data arises in many applications including smart environments, video surveillance, network analysis, human-robot interaction, and ambient assisted living. Our focus is on detecting behavior patterns that deviate from regular behaviors and might represent a security risk, health problem, or any other abnormal behavior contingency. In other words, deviant behavior is a data pattern that either does not conform to the expected behavior (anomalous behavior) or matches previously defined unwanted behavior (suspicious behavior). Deviant behavior patterns are also referred to as outliers, exceptions, peculiarities, surprise, misuse, etc. Such patterns occur relatively infrequently; however, when they do occur, their consequences can be quite dramatic, and often negatively so. Typical examples include credit card fraud detection, cyber-intrusions, and industrial damage.

This thesis targets a large class of problems with complex, spatio-temporal, sequential data generated by an entity capable of physical motion in environment, regardless of whether the observed entity is human, software agent, or even robot. In such domains, an agent often has an observable spatio-temporal structure, defined by the physical positions relative to static landmarks and other agents in environment. We suggest that this structure, along with temporal dependencies and patterns of sequentially executed actions, can be exploited to perform deviant behavior detection on traces of agent activities over time. Examples of such detection include: elderly persons, who are being monitored in their smart home and faces a gradual decrease in his health; a reckless driver zigzagging across two lanes; an attacker that tries to gain access at a high-security access point with a stolen identity; and a potentially suspicious passenger at the airport that appears to turn away in a presence of a police officer, but not blatantly so, hence no single observation is enough to raise a suspicion.

%Anomalies translate to significant (often critical) real life entities

%The main reason to perform deviant behavior detection is 



% Cyber intrusions
% Credit card fraud

% Credit Card Fraud
% An abnormally high purchase made on a credit card

% Cyber Intrusions
% A web server involved in ftp traffic


% Rare Class Mining
% Chance discovery
% Novelty Detection
% Exception Mining
% Noise Removal
% Black Swan*


% Defining a representative normal region is challenging
% The boundary between normal and outlying behavior is often not precise
% The exact notion of an outlier is different for different application domains
% Availability of labeled data for training/validation
% Malicious adversaries
% Data might contain noise
% Normal behavior keeps evolving

% spatio-temporal sequental data


% Supervised Anomaly Detection
% Labels available for both normal data and anomalies
% Similar to rare class mining
% Semi-supervised Anomaly Detection
% Labels available only for normal data
% Unsupervised Anomaly Detection
% No labels assumed
% Based on the assumption that anomalies are very rare compared to normal data


% Network intrusion detection
% Insurance / Credit card fraud detection
% Healthcare Informatics / Medical diagnostics
% Industrial Damage Detection
% Image Processing / Video surveillance 
% Novel Topic Detection in Text Mining



%
%==========================================================================================
%
\section{Problem Formulation}

%We leverage definitions by \cite{Sukthankar-AAAI2008} to formulate the problem. 

% Define plan recognition / keyhole plan recognition
The general problem of deviant behavior detection from an agent's sequential spatio-temporal traces is related to the problem of keyhole plan recognition. 
%
\index{agent}
We use the term \textbf{agent} to denote an intelligent, independent entity capable of physical motion and action, such as humans, simulated entities in virtual environments, or robots \citep{Sukthankar-AAAI2008}. 
%
\index{plan recognition}
\textbf{Plan recognition} refers to inferring the plan, or plans, of an intelligent agent from action observations in the environment~\citep{Schmidt1978}. % Charles F. Schmidt - http://www-rci.rutgers.edu/~cfs/472_html/Planning/PlanRecog.html
%
In \textbf{keyhole} \index{plan recognition!keyhole} plan recognition, the observed agent is unaware of, or indifferent to, being observed, whereas \textbf{intended} \index{plan recognition!intended} plan recognition assumes that the agent actively cooperates by choosing actions to make its intentions clear to the observer. 
By contrast, \textbf{obstructed} \index{plan recognition!obstructed} plan recognition assumes that the agent actively obstructs the plan recognition process \citep{Waern1995}.
Our work follows the assumptions of keyhole plan recognition, but it is not restricted to plan recognition only; instead, behavior is represented by patterns, as defined below.

%TODO: make a connection between keyhole plan recognition or remove it

\index{trace!spatio-temporal trace}
Agents are observed via \textbf{spatio-temporal traces}, a vector time series of the agent's physical positions and other sensor data describing the agent's state, such as inertial information, action, or activity. 
\index{behavior!agent's behavior}
Such vectors are used to determine \textbf{agent behavior}, a term that refers to the agent's responses to various perceptual inputs, whether those responses are overt or covert, and voluntary or involuntary. In other words, behavior is the range of actions and mannerisms made by an intelligent agent in conjunction with its environment, situation, and other agents. 

%Define actions, atomic activity, complex activity
\index{action}
\index{activity}
\index{activity!atomic activity}
\index{activity!complex activity}
From a complete set of observed spatio-temporal traces, we recognize and identify the following characteristics:
\begin{itemize}

\item \textbf{Actions and activities}: \index{action} \index{activity}
Actions and activities are defined as behavior primitives; that is, elements that help explain and describe the observed behavior of an agent in a specific time span.

\item \textbf{Behavior signature}: \index{behavior signature}
Agent behavior is presented in the form of \textbf{behavior signature}, such as a plan or pattern that encodes agent actions and responses to a situation over a period of time.

\item \textbf{Degree of deviation}: Behavior signature is compared to reference behavior signatures and expressed as a degree of deviation, which measures the likelihood that the observed behavior does not conform to the desired behavior.

\end{itemize}



\index{behavior!deviant behavior}
\index{behavior!suspicious behavior}
\index{behavior!anomalous behavior}
We use the term \textbf{deviant behavior} to denote agent behavior  that deviates from regular behavior of the same agent or other agents. There are two approaches to deviant behavior detection \citep{Avrahami-Zilberbrand2009}: \textit{suspicious} and \textit{anomalous} behavior detection. \textbf{Suspicious behavior} detection assumes a behavior library that encodes \textsl{negative} behavior signatures; that is, patterns are considered unwanted or suspicious as they correspond to an identifying match in the library. \textbf{Anomalous behavior} detection uses the behavior library in an inverse fashion, encoding only positive behavior signatures. When an observed behavior cannot be matched against the library, it is considered anomalous.

%
%==========================================================================================
%
\section{Challenges}

Deviant behavior detection is related to problems such as novelty detection~\citep{Markou03}, rare class mining~\citep{Elkan01}, chance discovery~\citep{Ohsawa2009}, exception mining~\citep{Luo2008}, and black swan events \citep{Taleb2007}. The common key challenges include defining a representative library of behavior signatures, the availability of labeled data for training/validation, dealing with noisy data, modeling normal behavior that keeps evolving, and different application domains' differing notions of an outlier. The class of problems tackled in this thesis, that is, problems with complex, spatio-temporal sequential data generated by an agent moving in a physical environment, poses several additional challenges. 

The first challenge is how to recognize atomic activities that constitute behavior patterns. Previous work in plan recognition assumes that atomic actions are either given or trivially obtained, while real-life applications require recognition from raw, and often multimodal, sensor readings. 

The second challenge is how to present complex, real-life behavior patterns that do not follow predefined scenarios. Presentation must be robust and flexible to describe sequential spatio-temporal traces compactly.

Third, deviant behavior may reflect on (i) different time scales, and (ii) different modalities. For example, an elderly person can quickly start limping after a minor stroke, which can be detected within hours with accelerometers attached to ankles, or can slowly start limping due to arthritis, which can be detected by comparing month-to-month behavior of daily activities (since the change is not significant for hourly comparison). The question is how to combine different time scales and modalities into a single evaluation. 

Finally, many domains include behavior where no single event is sufficient to decide whether behavior is deviant or not. 
%Consider a potentially suspicious passenger at the airport that turns away in a presence of a police officer. This single event on its own is not enough to raise a flag, since this can be simply a coincidence or noisy detection, but many such events put together cause the passenger to be threated as suspicious. 
There are three issues that need to be addressed. First, there is no single significant event or incident that would help to immediately reach a decision; rather the observed sequence is a series of observations that allow a decision. Second, there is no knowledge about the exact plans devised by the observed agent. Third, the behavior pattern's deviance degree depends on the past agent behavior. For example, a subsequent deviant pattern is evaluated differently than the first one, since the prior behavior indicates a tendency for deviant behavior. Hence, the simple counting of deviant patterns cannot be applied, since it accumulates all observations linearly. Furthermore, most of the plan recognition methods, which rely on a plan library, are insufficient, since plans are not known in advance.
%
Hence, an advanced approach is required to combine and accumulate deviation over time. 


%
%==========================================================================================
%
\section{Approach and Hypothesis}

There are four general evidence classes that are potentially valuable for deviant behavior detection: 
\begin{enumerate}
	\item spatio-temporal relationship of agent movement between landmarks fixed over a period of time,
	\item temporal dependencies between atomic actions in behavior patterns,
	\item time scales and modalities at which behavior patterns are processed, and
	\item behavior patterns that can be considered deviant when repeated.
\end{enumerate}

\vskip 0.5cm
\textbf{The hypothesis is that it is possible to leverage the available spatio-temporal cues, temporal dependencies, various time scales and modalities, and repetitive behavior patterns to detect anomalous and suspicious behavior.}
\vskip 0.5cm

\vskip 0.2cm
\noindent
\textbf{Spatio-temporal relationships and temporal dependencies}: Unlike the existing methodology, which tries to recognize exact or flexible behavior patterns or describe them, our proposed method focuses on activity dynamics and explores the relations between the spatial information and the activities. Spatio-temporal cues assume that the positive behavior patterns of the observed entity can be learned over time since they remain stable.

\vskip 0.2cm
\noindent
\textbf{Time scales and modalities at which behavior patterns are processed:} Most of the related work focuses on one specific viewpoint, be it in terms of time scale or sensor modality. Our main idea is to consider various aspects and hypotheses about a behavior pattern and the environment in order to construct a situational awareness and then, on this basis, make a reliable deviation estimation.

\vskip 0.2cm
\noindent
\textbf{Repetitive behavior patterns}: The main question addressed is how to combine multiple events to decide whether an event trace corresponds to normal or a deviant agent behavior. %T




%
%==========================================================================================
%
\section{Scientific Contributions}

This thesis led to the following original contributions:

\begin{enumerate}

	\item A unified anomalous and suspicious behavior detection framework,  incorporating the elements below, as well as demonstration on real-world domains.

	\item Problem definition and theoretical analysis of anomalous and suspicious behavior detection from agent traces, including optimality conditions and error bounds.

	\item New heuristic functions for detecting deviant agent behavior observed over longer periods of time where no significant event is sufficient to reach a decision.

	\item New representation of spatio-temporal behavior patterns that allows visual comparison of various patterns and can be efficiently deployed in anomaly detection algorithms.

	\item A comprehensive and flexible approach to activity recognition that  addresses sensor noise and activity mislabeling to provide activity primitives at various abstraction levels (that is, atomic activities and compound activities).

\end{enumerate}



%
%==========================================================================================
%
\section{Overview of the Thesis Structure}

This thesis comprises 11 chapters, organized in two parts as shown in Figure~\ref{fig:thesis-structure}. Chapter~\ref{chap:related_work} presents the background and surveys the related activity recognition work and anomalous and suspicious behavior detection. 

\begin{figure}[!ht]
\centering
\FramedBox{10cm}{\textwidth}{
Chapter \ref{chap:introduction}: \nameref{chap:introduction}

Chapter \ref{chap:related_work}: \nameref{chap:related_work}

	\leftskip=0.5cm 
	\FramedBox{5cm}{0.95 \textwidth}{
		\leftskip=0.5cm 
		Part \ref{part:theory}: \textbf{\nameref{part:theory}}
		
		\leftskip=1cm
		Chapter \ref{chap:activity_recognition}: \nameref{chap:activity_recognition}\\
		Chapter \ref{chap:signatures}: \nameref{chap:signatures}\\
		Chapter \ref{chap:detection}: \nameref{chap:detection}\\
		%Chapter \ref{chap:combine}: \nameref{chap:combine}\\
		Chapter \ref{chap:accumulation}: \nameref{chap:accumulation}\\
		Chapter \ref{chap:framework}: \nameref{chap:framework}
	}
	\vskip 0.5cm
	\leftskip=0.5cm 
	\FramedBox{3cm}{0.95 \textwidth}{
		\leftskip=0.5cm 
		Part \ref{part:applications}: \textbf{\nameref{part:applications}}
		
		\leftskip=1cm
		Chapter \ref{chap:confidence}: \nameref{chap:confidence}\\
		Chapter \ref{chap:lax}: \nameref{chap:lax}\\
		Chapter \ref{chap:civabis}: \nameref{chap:civabis}
	}
	\vskip 0.5cm

\leftskip=1cm 
Chapter \ref{chap:conclusions}: \nameref{chap:conclusions}
}
\caption{Thesis consists of 11 chapters structured in two parts.}
\label{fig:thesis-structure}
\end{figure}

Chapters~\ref{chap:activity_recognition}--\ref{chap:framework} constitute Part~\ref{part:theory} of the thesis, which gradually introduces components of the unified detection framework. Chapter~\ref{chap:activity_recognition} deals with activity recognition and introduces activity recognition pipeline as well as compound activity recognition and the recognition of agent-agent interactions. Chapter~\ref{chap:signatures} then presents the spatio-activity matrix approach to encode daily-living behavior patterns along with a visualization technique and a dimensionality reduction approach. Next, Chapter~\ref{chap:detection} establishes a formal detection framework, theoretically analyzes detection optimality and error bounds, and proposes several heuristics. Chapter~\ref{chap:accumulation} then further extends the framework to address the problem of repeated detection and proposes the F-UPR approach to accumulating suspicion over time. Finally, Chapter~\ref{chap:framework} connects all the components into a unified detection framework.

Chapters~\ref{chap:confidence}--\ref{chap:civabis} constitute Part~\ref{part:applications} of the thesis, which demonstrates how the framework is applied in three real-world domains. First,  Chapter~\ref{chap:confidence} focuses on the ambient assisted living domain, where the goal is to assess an elderly person's well-being to detect anomalies in daily-living patterns.
Second, Chapter~\ref{chap:lax} targets a class of applications where no single event is sufficient to determine whether behavior of an agent is suspicious or not; that is, suspicious passenger detection at an airport and dangerous driver detection. Third, in Chapter~\ref{chap:civabis}, the unified framework is utilized to improve security at a biometric access point using several modalities.

Finally, Chapter~\ref{chap:conclusions} summarizes the thesis, outlines the main contributions and discusses future work.


\section{Publications}
A number of previous publications underlie this thesis. The initial work on activity recognition was published by \cite{Lustrek2009Fall}. To address the challenges caused by sensor noise, \cite{Kaluza2009Glajenje} developed and published pre-processing filtering techniques, while \cite{Kaluza09Reducing} published the removal of spurious activity transitions (post-processing). The complete activity recognition pipeline was then fully applied and first published at the \textit{European Conference on Ambient Intelligence} \citep{Lustrek2009Behavior}. This publication enabled analyzing high-level behavior patterns such as spatio-activity matrices, which in turn was published at the \textit{International Conference on Machine Learning and Data Mining}~\citep{Kaluza2010ADL} and won the best student paper award. The paper was then further extended and published in \textit{Journal of Ambient Intelligence and Smart Enviroments} \citep{Kaluza2012:JAISE}.

The initial ideas for repeated anomalous and suspicious behavior detection's theoretical foundations were published at the PAIR workshop at the \textit{AAAI Conference on Artificial Intelligence} \citep{Kaluza2011:PAIR}, and then distilled along with the F-UPR detector as a full paper at the \textit{International Conference on Autonomous Agents and Multiagent Systems} \citep{Kaluza2012:AAMAS}.

Empirical studies on the ambient assisted living domain were also published at the \emph{International Joint Conference on Ambient Intelligence} \citep{Kaluza2010Agentbased} and demonstrated at the \textit{European Conference on Artificial Intelligence} \citep{Lustrek2012:ECAI} and the \textit{International Conference on Autonomous Agents and Multiagent Systems} \citep{Kaluza2012:AAMASdemo}. Results on the security domain were published in the \textit{Journal of Ambient Intelligence and Smart Environments} \citep{Dovgan2010jami} and \textit{Expert Systems with Applications} \citep{Kaluza2010:ESWA}. The comprehensive list of related publications is collected in Appendix~B.




