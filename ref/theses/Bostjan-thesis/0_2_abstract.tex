%!TEX root = thesis.tex

\begin{preliminary}{Abstract} 

\vspace{-1mm}

% Motivation
% - what is the domain
% - why is it interesting
% - why is it hard
\noindent
The problem of learning patterns of human behavior from sensor data arises in many applications including smart environments, surveillance, human-robot interaction, and ambient assisted living. Our focus of interest is detection of behavior patterns that deviate from regular behaviors and might represent a security risk, health problem or any other abnormal behavior contingency. Examples include a potentially suspicious passenger who appears to turn away at the presence of the police officer or an attacker that tries to access a high security access point with a stolen identity.

% Problem statement
% - what are the concrete problems
% - what are the challenges
% - how were addressed

Real-life applications for deviant behavior detection raise several challenges. The first challenge is how to recognize atomic activities that comprise behavior patterns. Previous work in plan recognition assumes that atomic actions are either given or trivially obtained, while real-life applications require recognition from raw sensor readings. The second challenge is how to present complex real-life behavior patterns that do not follow predefined scenarios. Presentation must be robust and flexible to describe unstructured real-life behavior patterns. Third, deviant behavior may reflect on different time scales and different modalities, hence the question is how to combine different time scales and modalities to a single evaluation. Finally, many domains include behavior where no single event is sufficient to decide whether behavior is deviant or not, hence an advanced approach is required to combine and accumulate deviation over time. 

% Our solution
% - what we propose
% - main contributions
This dissertation proposes a framework for analyzing agent's activities from prior knowledge and external observations to detect deviant behavior patterns, regardless of whether the observed entities are software agents, humans, or even robots. Specifically, the framework includes an algorithm for recognition of atomic activities, a novel presentation for complex real-life behavior patterns, an approach that combines different time scales and modalities, and a novel algorithm for accumulating deviation over time.

% Results
%  - experiments on domains
The framework is demonstrated on three domains: (i) behavior of persons at an access control point in high-security application; (ii) behavior of elderly persons that live at home alone in order to detect decreased behavior that indicates disease or deterioration in person's health; and (iii) suspicious passengers in the airport simulation. 


\end{preliminary} 

