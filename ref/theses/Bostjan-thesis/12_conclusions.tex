%!TEX root = thesis.tex

\chapter{Conclusions}
\label{chap:conclusions}

\section{Summary and Discussion}
This thesis addressed the problem of deviant behavior pattern detection within a large class of problems with complex, spatio-temporal, sequential data generated by an entity capable of physical motion in an environment.

The central scientific hypothesis of this thesis states that \textit{it is possible to leverage the available spatio-temporal cues, temporal dependencies, various time scales and modalities, and repetitive behavior patterns to detect anomalous and suspicious behavior.}

To this end, we developed new methods to extract spatio-temporal cues and temporal dependencies, and proposed a unified detection framework to address various viewpoints, as well as repetitive behavior patterns for anomalous and suspicious behavior detection.
%
To examine the validity of the hypothesis, we empirically demonstrated the unified detection framework on three domains.
%with results that generally outperformed the competing baselines. 
In the ambient assisted living domain, we demonstrated how to apply the framework to monitor an elderly person in a home environment to detect daily living anomalies, where the key component is an activity recognition pipeline and a spatio-activity matrix analysis. In the surveillance domain, we addressed the issue of repeated behavior detection and applied the framework to detect suspicious passengers at the airport. The novel F-UPR detector significantly outperformed the competing approaches. Finally, in the security domain, where the goal is to verify entering persons at a high-security access control point, we demonstrated a proof of concept of how the multimodal detection is beneficial. In summary, the thesis hypothesis is supported by the empirical evaluation and thus confirmed.% by the investigation within this thesis. 

\section{Scientific Contributions}

%We can now state the scientific contributions of this thesis, which closely match our initial expectations presented in the introduction. 
The work in this thesis has led to the following original contributions to science:
%This thesis addressed the problem of analyzing agent's activities from prior knowledge and external observations to detect deviant behavior patterns. We proposed a general framework for anomalous and suspicious behavior detection that accommodates the above-mentioned challenges. The main contributions are listed as follows:
\begin{enumerate}

	\item \textbf{A unified anomalous and suspicious behavior detection framework}:
	We proposed a unified framework for detection of anomalous and suspicious behavior that can be observed from complex, spatio-temporal sequential data generated by an agent moving in a physical environment. The framework incorporates several components to address the main challenges and is demonstrated empirically in three studies. 

	\item \textbf{Contribution to anomalous and suspicious behavior detection}: We gave the first clear problem definition and established a theoretical framework for anomalous and suspicious behavior detection from agent traces to show \textbf{how to optimally perform detection}. We discussed why detection error is often inevitable and \textbf{proved the lower error bound}. We further provided several heuristic approaches that either estimated distributions required to perform detection or directly rank the behavior signatures using machine-learning approaches.

	\item \textbf{Contribution to repeated behavior detection}: We extend the established theoretical framework and showed how to perform detection when an agent is observed over longer periods of time and no significant event is sufficient to reach decision. We first specified \textbf{conditions any reasonable detector should satisfy} and analyzed several detectors. We further proposed a novel approach denoted as \textbf{F-UPR detector} that generalizes utility-based plan recognition with arbitrary utility functions.
	
	\item \textbf{Contribution to behavior analysis}: We proposed a novel, efficient encoding denoted as \textbf{spatio-activity matrix} that is able to capture behavior dynamics in a specific time period using spatio-temporal features. We provided a visualization technique to compare different  behavior patterns. We further provided a feature extraction technique based on principal component analysis to reduce the spatio-activity matrix dimensionality, which can be directly used in anomaly detection algorithms.
	
	\item \textbf{Contribution to activity recognition}: 
	To address the problem of activity recognition from sensor data we introduced ARPipe, an \textbf{Activity Recognition Pipeline} that includes filtering, attribute construction, activity recognition, and activity smoothing. Within the pipeline, several novel algorithms were introduced including \textbf{body filter}, which applies human-body constraints to location-based body-attached sensors, and two approaches for \textbf{reducing spurious activity transitions} that cannot occur in reality are demonstrated.

	%\item \textbf{Unified detection framework}: TODO

	%\item \textbf{Empirical studies}: 

\end{enumerate}

\section{Future Work}

% key contributions

Anomalous and suspicious behavior patterns are rare, hence, a direction for future work is to consider approaches to expedite their appearance. For example, if the obtained deviation degree does not lead to confirmation, an observer might trigger an action toward the observed agent and observe its response to disambiguate his intentions, as it was demonstrated on an air-combat domain in a seminal work by \cite{Tambe95a}.

The unified framework proposed in this thesis has certain limitations in terms of deployment; for example, once the framework is trained and installed it does not take into consideration any feedback provided by the human operators behind it. One way to overcome this is to consider an online-learning mechanism that is able to incorporate human operator feedback in future behavior evaluations. Such a mechanism must not only adapt specific detectors to provide feedback, but also has to take into account gradual behavior drift of the agents interacting with(in) the environment. 
\cite{*}

% - activity recognition recursively applied on each level\\
% - formally analyze error bounds of specific methods\\
% - learning phase, online learning\\
% - actively trigger response from agent to determine behavior\\

