%!TEX root = thesis.tex

\chapter*{\vspace{-2.3cm} \Large Appendix D: Acknowledgments \vspace{1.7cm}}
\addcontentsline{toc}{chapter}{Appendix D: Acknowledgments}

\fancyhead[LO]{}
\fancyhead[RE]{Appendix D: Acknowledgments}

%This thesis would not have been possible without the generous help and support of my colleagues and family.

First of all, I would like to thank my supervisor Prof.\ Dr.\ Matja\v{z} Gams and co-supervisor Dr.\ Mitja Lu\v{s}trek. They provided me guidance, support, understanding, and professional assistance of the most valuable kind. Their extensive discussions and insightful explorations have been of great value for this work, which  could not have been carried out without them.

I am very grateful to my colleagues at the Department of Intelligent Systems at the Jo{\v z}ef Stefan Institute, several of whom deserve special merits. As regards the security domain, studied within the CIVaBiS project, I would like to thank the project team, in particular Tea Tu{\v s}ar, Erik Dovgan, Jana Krivec, Ale{\v s} Tav{\v c}ar, Robert Blatnik, as well as the security experts from the Slovenian Ministry of Defense, the {\v S}pica International d.o.o. team, who provided the hardware components, and Dr.\ Janez Per{\v s} from the Faculty of Electrical Engineering at the University of Ljubljana for computer vision expertise. 

The AAL domain experiments were made within the EU FP7 project Confidence. I am very thankful to the project team, in particular to Erik Dovgan, Violeta Mirchevska, Bo{\v z}idara Cvetkovi{\'c}, Rok Piltaver, Barbara Tvrdi, Bla{\v z} Strle, and colleges from the Department for Automation, Biocybernetics and Robotics at the Jo{\v z}ef Stefan Institute, as well as anonymous volunteers that made experimental recordings possible.   

I would like to express my deep gratitude to Prof.\ Dr.\ Milind Tambe for his encouragement, thoughtful guidance, support, and understanding during the course of research at the University of Southern California. I am also thankful to the members of the Teamcore research group for their useful discussions and friendly help. In particular, I would like to thank to Zhengyu Yin for valuable comments and suggestions regarding theoretical detection framework, Jason Tsai and Matthew Brown for their help with implementation challenges in the ESCAPES simulator, and James Pita, my office mate, for keeping me motivated. I am sincerely thankful to Prof.\ Gal Kaminka for constructive critisim and excellent insights. My warm thanks also go to Prof.\ Dr.\ Emma Bowring, Prof.\ Dr.\ Paul Scerri, Prof.\ Dr.\ Louis-Philippe Morency, and Prof.\ Dr.\ Ram Nevatia for providing valuable comments. 


I am also very thankful to Prof.\ Dr.\ Bogdan Filipi\v{c} and Prof.\ Dr.\ Marko Bohanec, who monitored my progress during the study and helped me to focus my research.

%I have to thank to my colleagues from Department of Intelligent Systems, several of whom deserve special merits. Bo\v{s}tjan Kalu\v{z}a shared with me many new valuable ideas and helped me to solve problems with software, especially problems with \LaTeX{}, which emerged in the last stages of writing this thesis. Damjan Ku\v{z}nar and Jana Krivec helped me in the early stages of the development of the HMDM method. The research on web genres presented in this thesis was built on the work done in collaboration with Dr.\ Mitja Lu\v{s}trek and Ale\v{s} Tav\v{c}ar. I am very thankful to Sanja Kova\v{c} who helped me to conduct experiments with genres.

%CIVABIS\\
%Tea, Erik, Ales, Jana, MG


I wish to thank to my family, to my wife Ajda, sister Mateja, and parents Damjana and Janko. They believed in me and were always there when I needed them.

I also want to thank to my classmates and good friends Ale\v{s} Tav{\v c}ar, {\v C}rt Gorup, Erik Dovgan, Dejan Petelin and Miha Mlakar, who helped me make my everyday commitments much easier, and Tony Guan, who made my Los Angeles experience exciting.

Last, but not the least, I am grateful to the Department of Intelligent Systems, Jo\v{z}ef Stefan Institute, and the Slovene Research Agency for providing me a scholarship, which made this thesis possible.
